\documentclass[12pt]{article}
\usepackage{microtype}
\usepackage{graphicx}
\usepackage{geometry}

\author{Tushaar Kamat}
\title{Musical Investigation Rough Draft}

\begin{document}
\maketitle

\section*{Rhythm}
\begin{itemize}
\item Both pieces share a duple meter, with a steady beat, although ``So What''
  is at a slightly faster tempo.
\item The melodic players in ``So What'' tend to stay slightly behind the rhythm
  section, while Shankar's playing is right on time with the drums and
  accompaniment. 
\item Both pieces use syncopation extensively, and the lead melodic players
  often use complex dotted quarter and eight notes in their improvisation, along
  with frequently playing on the upbeats instead of the downbeats. In the case
  of ``So What'', the eighth notes are actually extended into pairs of dotted
  eight notes and sixteenth notes, which creates a swing rhythm. 
\item Both pieces also shift in rhythm density, starting with a less dense
  opening riff, then evolving into more complex, denser riffs.
\end{itemize}
\section*{Dynamics}
\begin{itemize}
\item Both pieces' dynamics are largely controlled by the improviser. In ``Raag
  Khamaj'', Ravi Shankar frequently uses crescendos to build texture and emotion
  into his playing. 
\item The improvisers in ``So What'' and ``Raag Khamaj'' also employ sudden
  increases and decreases in dynamic. For example, Miles Davis, in his solo,
  begins with a series of muted, mellow riffs. However, this is interrupted by a
  sudden, loud sequence of 3 identical notes, which breaks the theme introduced
  earlier.
\item Both Davis and Shankar use explosive and crip attacks of the notes, in
  comparison to the much softer sustain and decay periods.  
\end{itemize}
\section*{Melody}
\begin{itemize}
\item Both songs alternate between legato and staccato playing, sometimes even
  consecutively for example, at the 6:20 mark in Raag Khamaj, Shankar repeats
  the same riff in staccato and legato style, one after the other. 
\item Both songs feature a ``head'' melody that is played at the beginning and
  and of the piece, but all of the other melody is improvised for both pieces.   
\item Ravi Shankar and the Davis Sextet, especially John Coltrain, frequently
  use scalar melodic constructions in their improvisation. Coltrain often
  precedes his riffs with runs up or down the Dorian scale, while Shankar often
  traverses the entire Raga, and both improvisers often end these riffs on the
  tonic. 
\item Both pieces have an extremely large range, and they traverse this range
  mostly in small steps, giving them a conjunct style. However, sometimes the
  improvisation will feature sudden jumps to high notes, to add drama. 
\item ``So What'' differs importantly from ``Raag Khamaj'' in that the main
  melody is actually in the bass line, as opposed to being in the sitar, which
  is the primary melodic instrument. 
\end{itemize}
\section*{Harmony}
\begin{itemize}
\item Both pieces feature an accompaniment section to provide harmony, although
  the rhythm section in the Davis sextet is much more complex than the tampura
  used to provide support to the sitar in ``Raag Khamaj''. 
\item Both pieces use harmonies that do not fit in the traditional major minor
  classification. ``Kind of Blue'' uses the Dorian mode, while ``Raag Khamaj''
  features semitones and a more complex modal scale known as a ``Raga''
\item Despite the tonality not being major or minor, the harmonies between
  instruments are generally consonant, with Evan's piano comping quickly
  shifting to match the current improviser's scale.
\item The accompaniment uses full chords for harmony in both pieces, with Evan's
  frequently hitting minor 7 chords to fit the Dorian scale, while the tanpura
  cycles through fragmented chords in ``Raag Khamaj,'' often shifting between
  the tonic and dominant.
\end{itemize}
\section*{Tone color}
\begin{itemize}
\item Both pieces feature a very high range, especially with the sitar in ``Raag
  Khamaj. However, even though the sitar has a greater range than any of the
  individual instruments in ``So What,'' the improvisation is split between 3
  different instruments with different ranges, making the range between pieces
  about equal.
\item ``So What'' has more instrumentation, with 3 different melodic instruments
  compared to only the sitar, allowing for greater tone color.
\end{itemize}
\section*{Texture}
\begin{itemize}
\item Both pieces feature homophonic style, with a single melodic instrument
  playing over the accompanying harmonic instruments.
\item While both pieces do feature harmonies, they lack more complex harmonic
  structures in comparison to Western Classical music, such as counterpoint.
  However, while both pieces are not extremely developed in texture, they make
  up for this with intricate improvised melodies that constantly surprise the
  listener.
\end{itemize}
\section*{Form}
\begin{itemize}
\item Both pieces follow a head - improvisation - head format, and both pieces
  do not fit into any of the traditional classical forms such as binary,
  ternary, etc.
\item Both pieces are completely instrumental, and do not feature any vocals or
  lyrics.
\end{itemize}


\end{document}
